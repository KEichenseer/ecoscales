% Use only LaTeX2e, calling the article.cls class and 12-point type.

\documentclass[12pt]{article}

% Users of the {thebibliography} environment or BibTeX should use the
% scicite.sty package, downloadable from *Science* at
% www.sciencemag.org/about/authors/prep/TeX_help/ .
% This package should properly format in-text
% reference calls and reference-list numbers.

\usepackage[sort]{scicite}
\usepackage{times}
\usepackage{graphicx}
\usepackage{lineno}
\usepackage{color}

% The following parameters seem to provide a reasonable page setup.

\topmargin 0.0cm
\oddsidemargin 0.2cm
\textwidth 16.5cm 
\textheight 21cm
\footskip 1.0cm


%The next command sets up an environment for the abstract to your paper.

\newenvironment{sciabstract}{%
\begin{quote} \bf}
{\end{quote}}


% If your reference list includes text notes as well as references,
% include the following line; otherwise, comment it out.

%\renewcommand\refname{References and Notes}

% The following lines set up an environment for the last note in the
% reference list, which commonly includes acknowledgments of funding,
% help, etc.  It's intended for users of BibTeX or the {thebibliography}
% environment.  Users who are hand-coding their references at the end
% using a list environment such as {enumerate} can simply add another
% item at the end, and it will be numbered automatically.

\newcounter{lastnote}
\newenvironment{scilastnote}{%
\setcounter{lastnote}{\value{enumiv}}%
\addtocounter{lastnote}{+1}%
\begin{list}%
{\arabic{lastnote}.}
{\setlength{\leftmargin}{.3in}}
{\setlength{\labelsep}{.5em}}}
{\end{list}}


% Include your paper's title here

\title{The Spatial and Temporal Domains of Modern Ecology }%\\
%OR Ecology's Spatio-temporal Domains\\
%OR Space, Time, and Ecology \\
%OR Ecology: still a small-scale, field-based discipline} 
 
\author
{Lyndon Estes$^{\ast1, 2, 3}$, Paul R. Elsen$^{1, 4}$, Tim Treuer$^{5}$, Labeeb Ahmed$^{6}$, \\
Kelly Caylor$^{1, 7}$, Jason Chang$^{6}$, Jonathan J. Choi$^{5}$, and Erle Ellis$^{6}$ \\
\\
\normalsize{$^{1}$Woodrow Wilson School, Princeton University, Princeton, NJ 08544, USA}\\
\normalsize{$^{2}$Civil and Environmental Engineering, Princeton University, Princeton, NJ 08544, USA}\\
\normalsize{$^{3}$Graduate School of Geography, Clark University, Worcester MA 01610, USA}\\
\normalsize{$^{4}$Department of Environmental Science, Policy, and Management,}\\
\normalsize{University of California Berkeley, Berkeley, CA 94720, USA}\\
\normalsize{$^{5}$Ecology and Evolutionary Biology, Princeton University, Princeton, NJ 08544, USA}\\
\normalsize{$^{6}$Geography and Environmental Systems, University of Maryland Baltimore County,}\\ 
\normalsize{Baltimore, MD 21250, USA}\\
\normalsize{$^{7}$Earth Research Institute, University of California Santa Barbara, Santa Barbara, CA 93106, USA}\\
\\
\normalsize{$^\ast$To whom correspondence should be addressed; E-mail:  LEstes@clarku.edu.}}
% Include the date command, but leave its argument blank.

\date{}



%%%%%%%%%%%%%%%%% END OF PREAMBLE %%%%%%%%%%%%%%%%



\begin{document} 

% Double-space the manuscript.

\baselineskip24pt

% Make the title.

\maketitle 



% Place your abstract within the special {sciabstract} environment.

%\begin{sciabstract}
\section*{Abstract}
%Ecologists have long known that the observational scales play a critical role in shaping our understanding of ecosystems, and that observations spanning multiple scales of space and time are needed to effectively characterize ecological phenomena. 
\textbf{In order to properly understand ecological phenomena, it is necessary to quantify their behavior over the range of spatial and temporal scales at which they manifest. Ecology has been concerned with this need since the early 1990s, and the ability to collect multi-scaled ecological observations has grown rapidly since then.  Characterizing the spatial and temporal domains of modern ecological observations can therefore provide important insight into the field's progress in understanding towards a more comprehensive understanding of ecosystem behaviour. To characterize these domains, we conducted a meta-analysis of recent (2004-2014) ecological studies, in which we quantified four primary dimensions of their reported observations: plot resolution, sampling interval, effective duration (time between start and end of temporal replicates), and effective extent (area enclosed by spatial replicates). We also estimated the \emph{actual} extent and duration, which respectively represent the summed area and time covered by spatial and temporal replicates. \textcolor{blue}{Replace this text with more specific summary of observed scales:} \textcolor{red}{Here we show that ecology remains a largely field-based discipline that makes observations within generally narrow spatial and temporal domains, despite the well-established literature on the importance of scale \cite{levin_problem_1992,chave_problem_2013,wiens_spatial_1989}}}. % Ecologists still make limited use of remote sensing (6\% of observations), which is necessary for observing larger-scale patterns and processes \cite{turner_remote_2003,pettorelli_satellite_2014}, but make slightly greater use of technologies that capture high frequency or temporally continuous point-based measurements (12\% of observations). Resolution, extent, and duration were not clearly reported in most reviewed studies. This indicates inattentiveness to scale, which, together with the narrow spatio-temporal domains of observations, suggests a limited ability to generalize the knowledge gained in many ecological studies, which is critical for developing an ability to predict ecological responses in an era of increasing global change \cite{levin_problem_1992}. An emerging wave of observing and analytical technologies (e.g. nano-satellites, unmanned systems, open image archives, and high-power processing platforms) now makes it dramatically easier and cheaper to collect high-quality, ecologically meaningful measurements at scales inaccessible to field scientists. Combined with a greater attention to scale in research design, these advances can facilitate the spread of ecological research to sparsely observed scales, while improving the transferability of ecological knowledge.}

%Observational scales play a critical role in shaping our understanding of ecological pattern and process. To gain insight into the spatial and temporal domains being addressed by modern ecological science, we conducted a meta-analysis of papers published in 30 ecological journals between 2004 and 2014. We estimated two spatial dimensions (resolution and extent) and two temporal dimensions (interval and duration) of 371 ecological observations reported within 140 papers (out of 346 randomly selected titles). Our analysis revealed that ecology largely remains a field-based discipline that primarily collects observations within narrow spatial and temporal domains. Ecologists still make limited use of methods for observing larger-scale patterns and processes (e.g. remote sensing, used for 6\% of observations), but make greater use of those which permit high frequency or temporally continuous \emph{in situ} measurements (12\% of observations). Resolution, extent, and duration were not clearly reported for the majority of observations, which, together with the narrow spatio-temporal ranges of observed scales, suggests a limited ability to generalize the findings of many studies.    

%  An example of the style is the special\texttt{\{sciabstract\}} environment used to set up the abstract you see here.
%\end{sciabstract}

\linenumbers
\vspace{10pt}
The scales at which ecosystems are observed plays a critical role in shaping our understanding of their structure and function \cite{levin_problem_1992,chave_problem_2013,wiens_spatial_1989}.  Ecological patterns emerge from temporal and spatial domains that may be coarser or finer than the processes that shape them, which means that investigation across multiple scales is the \emph{sine qua non} for understanding ecological phenomena \cite{levin_problem_1992}. This awareness has grown rapidly since the 1980s, accelerated by the need to understand how changes in the global climate, ocean, and land systems are affecting everything from individual populations \cite{tingley_push_2012} to entire biomes \cite{xiao_photosynthetic_2004}, while technological advances in areas such as remote sensing and genetics are making it ever-easier to quantify ecological features across a broad range of scales \cite{schneider_rise_2001, chave_problem_2013}.  

Given this awareness of the centrality of scale to understanding ecology, and the growing ability to study ecological phenomena across a broad range of spatial and temporal scales, it is important to assess the scales at which current ecological research is conducted. To gain insight into this question, we quantified the spatial and temporal domains of empirical observations that were reported in a representative sample of studies published between 2004-2014 in the top 30 ecological journals (by impact factor). Empirical observations provide the necessary means for developing and testing the models that explain why ecological patterns vary in time and space \cite{levin_problem_1992, tilman_ecological_1989}, thus it stands to reason that the temporal and spatial distributions of ecological observations may be indicative of modern ecology's progress towards achieving a holistic, predictive understanding of ecosystems \cite{chave_problem_2013,levin_problem_1992}. 

We characterized observational domains along two key spatial dimensions, resolution (grain) and extent, and their temporal corollaries, interval and duration (Table 1).  Here resolution is the area of an individual spatial replicate, or the two-dimensional space in which all measurable features of a natural object(s) were recorded (as opposed to sub-sampled), while extent is the area enclosed by the outer-most spatial replicates, or, if the system or habitat being sampled was distinct from its surrounding matrix (e.g. forest patches in grassland habitats), the summed area of sampled patches (see SI for full definition). Interval refers to the average time elapsed between individual temporal replicates, and duration is the time elapsed between the first and last temporal replicates, or, in the case of temporally unreplicated observations, the estimated time spent collecting the observation (SI). We also calculated two additional metrics, the integrated area of spatial replicates (i.e. resolution multiplied by number of replicates) and the summed observational time of all temporal replicates. We estimated these additional dimensions to evaluate the degree to which the actual scales of ecological observations differ from those they ostensibly represent, and therefore refer to them as the \emph{actual} extent and \emph{actual} duration.

\begin{table}[htp]
\caption{The dimensions of ecological observations estimated in this meta-analysis.}
\begin{center}
\begin{tabular}{ll}
\hline
\textbf{Dimension} & \textbf{Description} \\
\hline
Resolution & Area (m$^2$) of an individual spatial replicate (plot) \\
Extent & Area (ha) encompassed by all spatial replicates \\
Actual extent & Summed area (ha) of all spatial replicates\\
Interval & Time elapsed (days) between successive temporal replicates \\
Duration & Time elapsed (days) between first and last temporal replicates\\
Actual duration & Summed observational time (days) of all temporal replicates\\
\hline
\end{tabular}
\end{center}
\label{default}
\end{table}%

We calculated these dimensions from 379 discrete observations reported within a 134 paper subset of 348 randomly selected articles (from 42,918 total). An additional 62 papers that were cited as the source of observations in the selected papers were also reviewed. We confined our analysis to observations made of ``natural'' \cite{tilman_ecological_1989}, or non-experimentally manipulated, systems, given that the inclusion of experiments could have skewed our assessments towards the relatively fine scales at which such studies typically focus. 

To account for uncertainty in the estimation of observational dimensions due to 1) unclear methodological description in the reviewed papers, and 2) observer interpretation, we conducted a resampling analysis (n=1000) in which scale values were randomly perturbed within the bounds of estimated inter-observer variation (SI). We constructed histograms for each dimension from the mean of the perturbed ensembles, and estimated 95\% confidence intervals for each histogram bin (Fig. 1). We constructed kernel density estimates from the full resampled ensemble in order to assess observational distributions within different juxtapositions of the four space-time dimensions (Fig. 2). 

In terms of resolution, the majority (67\%) of observations were collected in plots of $<$1 m$^2$ resolution, 24\% were collected within plots of 1 m$^2$ up to 1 ha, and the remaining 9\% in plots of $\geq$1 ha (Fig. 1A). The extent of 19\% of observations was $<$10 ha, 23\% covered 10-1,000 ha, 42\% 1000-1,000,000 ha, and 16\% $>$1,000,000 ha (Fig. 1B). 

%16 + 29 + 26 + 67 + 14 + 37 + 10

\begin{figure}[!ht]
%\begin{wrapfigure}{c}{1\textwidth}
\includegraphics[width=1\textwidth]{figures/hists_bs.pdf}
\vspace{-0.15 cm}
\caption{Histograms of the resolution (A), extent (B), interval (C), and duration (D) of observations collected from the surveyed ecological studies. Bars represent the average percentages for each bin realized after 1000 perturbed resamples, while grey bars indicate the 95\% confidence interval. }
\label{afoto1}
\end{figure}

In the temporal dimensions, 37\% of observations were not repeated (Fig. 1C), 17\% were repeated at short intervals (sub-second to daily), 20\% at daily to monthly intervals, 18\% at monthly to yearly intervals, and 8\% at yearly to decadal intervals. Duration was one day or less for 31\% of sampled observations, while 10\% covered one day to one month, 23\% lasted one month to one year, 27\% covered 1-10 years, and 10\% spanned a decade or more (including several paleoecological studies covering centuries to millennia; Fig. 1D).

Juxtaposing these observational dimensions provides further insight into the spatio-temporal distribution of ecological observations and the domains in which they are concentrated (Fig. 2). Contrasting resolution with interval reveals that the majority of temporally replicated observations (the 37\% that were unreplicated were excluded because they lack interval values) had resolutions of 10 cm$^2$-1 m$^2$ and were revisited at daily to yearly intervals (Fig. 2A). A less dense, oblong concentration of observations bounded on the lower right by monthly to yearly observations at 100 m$^2$ and on the upper left by near-daily to monthly observations with 1-10 ha resolution is also evident. This lower right to upper left orientation reflects the tradeoff between resolution and interval that is typical of satellite imaging \cite{estes_platform_2016}, and stands in contrast to the upper right to lower left line that stretches between this concentration and the high frequency (minute-hour intervals), high spatial resolution (0.1-100 cm$^2$) observations. This line demonstrates the opposite tradeoff that occurs with field-based observations, where larger plot sizes demand greater effort that in turn reduces sampling frequency \cite{kareiva_spatial_1988}.   

%Comparing the spatial and temporal resolutions of  interval to spatial resolution of ecological observations reveals three primary regions of observational concentration (Fig. 2A), the greatest being unreplicated samples with resolutions of 100 cm$^2$ to 1 m$^2$, followed by observations of daily to monthly interval with slightly finer spatial resolutions (10 cm$^2$ to 1 m$^2$), and finally a smaller concentration of monthly to yearly measurements collected at spatial scales of 0.1-10 ha.  

\begin{figure}[!ht]
%\begin{wrapfigure}{c}{1\textwidth}
\includegraphics[width=1\textwidth]{figures/kde45.pdf}
\vspace{-0.15 cm}
\caption{Two-dimensional kernel density estimates of observational densities within the domains defined by A) interval and resolution (of temporally replicated observations only), B) duration and extent, C) resolution and extent, and D) interval and duration (of temporally replicated observations). Density estimates were applied to the log-transformed values of each observational dimension, and density estimates are rescaled to represent percentages.}
\label{afoto1}
\end{figure}

Contrasting duration and extent (for all observations) reveals two primary domains of observational concentration. The first consists of observations spanning one month to one decade in time and 10-1000 ha in space, while the second is defined by observations of one year to several decades that cover 10,000 to 1,000,000 ha (Fig. 2B).  Three other notable, but lesser areas of concentration are also evident, including small area observations (0.1-1 ha) covering one month to decade, and short duration, temporally unreplicated observations ($<$1 day) of either 1-10 ha or 10,000-1,000,000 ha. 

% the bulk of ecological observations span one month to one decade in time and covered total time periods of one minute to $<$1 month and areas of $\leq$1 m$^2$ (Fig. 2B). The next greatest concentration were instantaneous ($\leq$1 second) observations of $<$1 m$^2$ extent, followed by a fainter concentration centered on 1 hour duration and between 1 m$^2$ and 1 ha in extent. Very few observations had spatial extents $>$1,000,000 ha, and because these were collected by satellites that make near-instantaneous measurements, they tended to have aggregate durations of $\leq$1 hour. In contrast, the longest periods of observation were provided by paleo-ecological or long-term weather stations, both of which produce continuous, point-scale measurements (e.g. sediment cores and weather instrumentation). 

Comparing the two spatial dimensions against one another (for all observations) shows a primary concentration of observations with 10 cm$^2$ to 100 m$^2$ resolution that have extents ranging between slightly over 1,000 to nearly 1,000,000 ha (Fig. 2C). The second-most prominent concentration consists of higher resolution (1 cm$^2$-1 m$^2$), smaller extent (10-1000 ha) observations, beneath which lies a third and fainter concentration of 1-1,000 cm$^2$ resolution, 1000 m$^2$ to $<$10 ha.  These three concentrations suggest a tendency for observational extent to increase with resolution, which is a relationship that becomes more pronounced in the (less densely observed) portion of the domain where resolutions $\geq$100 m$^2$.  

A similar tendency for duration to increase with interval was also evident amongst temporally replicated observations (Fig. 2D), where the majority of observations were repeated at daily to decadal intervals and spanned  $\geq$ 1 month to $<$100 years.  The orientation of this concentration shows that observations lasting one year to one decade tend to have corresponding intervals, while those lasting one month to one year have daily to monthly intervals. This suggests that most long-duration observations have just a single temporal replicate. The low densities of observations having sub-daily intervals shows that relatively few high frequency, long duration ecological measurements are undertaken.  

To provide further insight into the spatio-temporal domains of ecological observations, we also evaluated the differences in scale between extent and actual extent and between duration and actual duration. To make these comparisons, we calculated the magnitudes of difference (decades) between each dimension and its actual equivalent, and examined how these varied as a function of the scale of the actual measurement (Fig. 3). The majority (80\%) of the assessed observations had actual extents of $\leq$1 ha, which on average was 4 to nearly 8 orders of magnitude smaller than the quantified extent (Fig 3.A).  Actual extent converged with extent above 100,000 ha, but this applies to just 5\% of observations. The actual duration of 65\% of observations is $<$ 1 day, which is on average 3 to nearly 9 orders of magnitude shorter than the time span covered by temporal replicates (Fig. 3B). The two duration measures were effectively the same for the 16\% of observations exceeding one month of actual duration. 

\begin{figure}[ht]
\includegraphics[width=1\textwidth]{figures/act_v_eff_diff.png}
\vspace{-0.2 cm}
\caption{The difference between \emph{actual} extent (the summed area of spatial replicates) and extent (A) and \emph{actual} duration (the summed sampling duration across temporal replicates) and duration (B).  Differences are expressed as decades, or how many orders of magnitude greater the extent or duration is then the actual extent or duration, and are summarized (as box plots, with circle in box representing the mean and line the median) in bins representing different levels of actual extent/duration.  The percentages of observations falling within each bin are indicated by the color of the inter-quartile and the numeric value above the upper whisker.}
\label{afoto1}
\end{figure}


\noindent \textbf{\emph{Observational methods}}\\
We classified the method used to collect each observation into several broad categories, which were field methods (manual \emph{in situ} data collection), automated (\emph{in situ}) sensing, remote sensing, other geographic data, and paleo-reconstruction approaches. Field methods were used for 80\% of observations, automated sensing for 12.4\%, remote sensing for 6.3\%, and paleo-reconstruction and other geographic data each for less than 1\%. Using linear regression (weighted by the number of observations per publication year) to assess whether the relative frequency of observing methods changed during the 10 year study period, the use of remote sensing appeared to increase by 1.3\% per year from 2004-2014 (R$^2$ = 0.25, p$<$0.12), and field methods declined by the same percentage (R$^2$ = 0.1, p$<$0.18), although both slopes failed to meet the customary threshold for statistical significance. Automated sensing methods showed no trend (SI). 

\noindent \textbf{\emph{Potential biases and uncertainties in quantifying observational scales}}\\
%This analysis did not necessarily quantify the scales \emph{represented} by ecological observations (SI). Observations that were non-continuous in time or space (e.g. point-based field measurements) may effectively represent larger scales than our estimates suggest, due to phenomenon-dependent factors such as autocorrelation and representativeness of the sampling scheme \cite{underwood_experiments_1997, palmer_scale_1994,cao_comparison_2002, legendre_spatial_1993,collins_method_2000-1}. From a spatial perspective, this concern does not apply to contiguous sampling schemes, which were primarily based on remote sensing and cover the largest area, thus the net effect is that this analysis may underestimate the effective areas of smaller observational extents. From a temporal perspective, our analysis likely underestimates the effective duration of many observations, particularly those where instantaneous, repeated measures of slow-changing ecological features were made. Snapshots in time may be sufficient to capture the temporal dynamics of such phenomena (e.g. changes in vegetation cover \cite{hansen_high-resolution_2013}), and for these the total time elapsed between the first and last measurements (the span of observation) may more closely approximate effective duration. However, many studies focused on more dynamic phenomena, and for these long periods of continuous observation may be more important for understanding dynamics than frequent repeats. For example, wildfire extent and duration can be mapped by daily return satellites \cite{roy_prototyping_2005,jones_fire_2009}, but the instantaneous nature of the imaging means that it cannot be used to directly measure fire behavior \cite{clements_observing_2007}, except patterns resolving themselves at scales greater than 12-24 hours. For this reason, and to provide a consistent standard for estimating duration, we recorded actual duration. 

%Adding to our motivation to record actual observational scales was the fact that 
There were several potential methodological aspects that could have influenced our assessment of ecology's spatial and temporal domains. The first stems from our finding that many studies did not precisely report observational scales, which meant that we had to estimate, rather than simply record, these values for most observations (specifically, in 63, 60, and 67\% of cases for resolution, extent, and actual extent, and 36\%, 64\%, and 83\% of cases for interval, duration, and effective duration). The inevitable estimation errors may have biased our overall findings. However, we attempted to quantify this error by assessing inter-rater disagreement and resampling techniques. The resulting confidence intervals (Fig. 1) suggest that it was unlikely that estimation errors unduly influenced our findings. 

Another potential source of bias was the rule set we used to estimate observational scales, chiefly through our definition of resolution as the smallest area in which all features of interest were measured. We made this choice because we wanted to use a consistent standard. However, some papers reported plot resolutions wherein sub-samples were made within the plot area, thus the average resolution (and actual extent) calculated from our estimates is likely to be smaller than it would be if we had simply recorded reported resolution in all such cases. In addition to this, our decision to exclude experimentally manipulated observations is also likely to have influenced the distributions of all assessed dimensions. For example, the average values of estimated resolutions would likely be smaller if we had included experiments.  

Finally, because our review did not include papers beyond 2014, the omission of studies from the most recent years could have introduced bias into our domain estimates. Indeed, if the trend towards increasing use of remote sensing between 2004-2014 was not spurious, we can project that a repeated study applied to papers published between 2004-2017 would find that remote sensing was used for 7.7\% of observations (a 22\% increase), which would increase the mean extent by 17.4\% (95\% CI = -1.3-67\%), or 0.07 orders of magnitude, above that observed here (see SI for details of calculation). Further support for such a trend can be found by averaging dimensions by publication year (SI), which reveals that observational extent increased by 0.253 orders of magnitude per year between 2004-2014 (R$^2$ = 0.2.5, p$<$0.07). This somewhat clearer trend also suggests that including more recent studies would show extent to be somewhat larger, although in this case by a more modest 5.5\% (0.02 orders of magnitude).  


%Additional sources of uncertainty: rules for extracting scales; diversity of study types and ecological methods. 

%Most studies provided insufficient methodological detail to judge whether sampling schemes were suitable matched to and representative of the phenomena they were measuring. The observational scales were not clearly reported in many studies, such that we had to estimate spatial resolution and extent in 62\% and 68\% of cases, respectively, and temporal interval and duration in 40\% and 86\% of cases. Although the confidence intervals around our scale estimates (Fig. 1) suggests that our findings are robust to this level of uncertainty, it would have likely confounded efforts to estimate effective scales. 

\noindent \textbf{\emph{Insights into the scales of modern ecology}}\\
The continued dominance of field-based research, and the limited use of methods that allow larger areas to be comprehensively observed, such as remote sensing.

%These results indicate that for most ecological observations are collected with spatially discontinuous replicates that sample very small portions of the area in which they are distributed.


Satellite observations typically have lower information content than field measurements for a given location, and in many cases only provide proxy measures for the ecological features of interest, such as forest understorey structure \cite{estes_remote_2010}, thereby making ecologists less inclined to use the technology \cite{turner_remote_2003}.

Adapt and tone down, retain these points:
Furthermore, the unclear documenting of observational scales implies that scale is not a primary concern in much ecological research \cite{chave_problem_2013, wheatley_factors_2009}. (Scale is not a concern that cuts across the entire discipline of ecology). 
Narrowness and poor documentation (a tendency that is also evident in the geographical sciences \cite{margulies_ambiguous_2016}) 
Ecological understanding drawn from many of these observations may have limited generalizability \cite{margulies_ambiguous_2016, wheatley_factors_2009, wiens_spatial_1989}, 
a concern that has been previously noted due to ecology's geographical bias towards anthropogenically undisturbed and temperate ecosystems \cite{martin_mapping_2012}. 



Our results provide valuable insight into the spatial and temporal domains being addressed by modern ecological research. Our results show that most observations are collected at small spatial scales, are either unrepeated or relatively infrequent ($\geq$1 month interval), and in aggregate cover relatively narrow periods of time ($\leq$1 month). Very little research is conducted at high spatial and temporal resolutions over large areas or for long time periods, indicating that, despite the well-established understanding of the importance of multi-scale assessments for understanding ecological patterns and processes \cite{levin_problem_1992,wiens_spatial_1989}, efforts focused on larger scales are still relatively sparse within the discipline \cite{levin_problem_1992,wiens_spatial_1989}. Furthermore, the unclear documenting of observational scales implies that scale is not a primary concern in much ecological research \cite{chave_problem_2013, wheatley_factors_2009}. Taken together, this narrowness and poor documentation (a tendency that is also evident in the geographical sciences \cite{margulies_ambiguous_2016}) suggests that the ecological understanding drawn from many of these observations may have limited generalizability \cite{margulies_ambiguous_2016, wheatley_factors_2009, wiens_spatial_1989}, a concern that has been previously noted due to ecology's geographical bias towards anthropogenically undisturbed and temperate ecosystems \cite{martin_mapping_2012}. 

The generally small spatial scales of observation is a consequence of the continued dominance of field-based research, and the limited use of methods that allow larger areas to be comprehensively observed, such as remote sensing. Despite early and repeated calls for ecologists to use remote sensing because it provides a synoptic view that field measurements cannot \cite{turner_remote_2003, kerr_space_2003, pettorelli_satellite_2014}, and subsequent demonstration of its importance for multi-scale studies \cite{estes_habitat_2008, estes_predictive_2011}, our results indicate this method has not yet been widely adopted in ecological research. Two reasons lie behind this slow uptake. First, remote sensing can be a challenging method for ecologists to learn, many of whom may not have access to appropriate training \cite{pettorelli_satellite_2014}. Second, satellite observations typically have lower information content than field measurements for a given location, and in many cases only provide proxy measures for the ecological features of interest, such as forest understorey structure \cite{estes_remote_2010}, thereby making ecologists less inclined to use the technology \cite{turner_remote_2003}. 

In contrast to remote sensing, ecologists have made broader use of technologies that increase the temporal resolution of observations. Automated sensors were used to record 12\% of all observations, and accounted for most very high frequency measurements (intervals $\leq$1 hour). As with spatial data, finely resolved temporal data can be aggregated to facilitate multi-scale analyses (and many of the reviewed studies that used automated sensing aggregated the resulting observations before analyzing them), whereas longer-interval data (e.g. annual biomass accumulation) cannot be disaggregated into shorter interval measurements (e.g. weekly biomass accumulation) without interpolation, which turns data into modeled, rather than direct, observations. 

In the coming years, rapid technological advances should increase the concentration of ecological observations in currently under-represented domains. The growing numbers of high-resolution satellite sensors, together with new analytical platforms that provide free access to large volumes of pre-processed data and computational power \cite{googleearthengine}, will lower technical barriers that have so far prevented ecologists from adopting this observational technology  \cite{pettorelli_satellite_2014}. Similarly, the advent of unmanned systems offers the ability to measure ecological features at high spatial and temporal frequencies over large areas \cite{anderson_lightweight_2013}, which were scales that were previously impractical to access. The ever-falling cost of sensor technology and the ubiquity of cell phones also means that ecologists, together with a growing army of citizen-scientists, have the unprecedented ability to make spatially dense, high frequency observations over large areas \cite{wolf_gsm-based_2012,collins_new_2006,porter_wireless_2005,dickinson_current_2012}.  A greater attentiveness to scale in general, including more meticulous documentation of observed dimensions, may help to facilitate the spread of ecological research to sparsely studied scales, while improving transferability of knowledge within the discipline.  

%Plot resolution issue: need high resolution over large extents, because this relies on remote sensing, and to really capture details, need high resolution.  
%likely because of perceived (and in some cases real) logistical barriers \cite{pettorelli_satellite_2014}. 

%These findings suggest that ecological understanding is shaped by studies of relatively fine-scaled patterns, and because they are small in extent may be less generalizable \cite{wiens_spatial_1989}.   

%Despite the understanding that ecological patterns and process change with scale \cite{levin_problem_1992}, the narrow distribution of observations within spatial domains calls into question the degree to which ecological phenomena are being investigated at multiple scales. Adding to this concern is the fact the scales of observation were not clearly reported in the majority of studies; the spatial resolution and extent of studies was uncertain in 64\% and 70\% of cases, respectively, while the precise interval and study duration were uncertain in 43\% and 86\% of cases. Although sensitivity analyses show that our conclusions are robust to this uncertainty (SI), the unclear documenting of observational scales suggest that they are not a primary methodological concern in ecological research \cite{chave_problem_2013, wheatley_factors_2009}. 
%
%On the other hand, we found a greater diversity of scales in the temporal domains of observations (Fig. 1D). Over 70\% of sampled observations involved repeated assessments, and one-quarter of these were made with high frequency (intervals $\leq$1 day). Automated, surface-based sensors were used to record most of the very high frequency observations (intervals $\leq$1 hour), and comprised 9\% of all observations, suggesting that ecologists are adopting new technologies for collecting temporal data more rapidly than they are for spatial data.  As with spatial data, finely resolved temporal data can be aggregated to facilitate multi-scale analyses, whereas it is extremely difficult, if not impossible, to disaggregate long-interval data. 
%
%The temporal duration of most collected observations appears to be brief ($\leq$1year), which is due to the fact that we estimated the summed time of repeated observations, rather than the total time elapsed between the first and last measurements (the span of observation). In many cases, the actual observations are very brief, and thus the temporal duration is narrow. Snapshots in time may be sufficient to capture the temporal dynamics of many phenomena (e.g. changes in vegetation cover \cite{hansen_high-resolution_2013}), and for these our measurement of this domain may understate the temporal breadth of ecological observations. However, for more dynamic phenomena, long periods of continuous observation may be more important for understanding dynamics than frequent repeats. For example, wildfire extent and duration can be mapped by daily return satellites \cite{roy_prototyping_2005,jones_fire_2009}, but the instantaneous nature of the imaging means that it cannot be used to observe behavior and evolution of the active fire \cite{clements_observing_2007}. For this reason, and because duration is the temporal analog of spatial extent, we chose to record observational duration rather then span. 

%Based on the observational domains recorded here, it appears that ecologists are still primarily collecting data at very fine spatial scales and over very narrow extents. Although then temporal duration of most ecological observations is also narrow, and a sizable number of observations are not repeated, greater effort is being devoted to capturing temporal dynamics. These findings suggest that ecological understanding is shaped by studies of relatively fine-scaled patterns, and because they are small in extent may be less generalizable \cite{wiens_spatial_1989}.   


%Based on this analysis, ecology thus appears to have

%High frequency, high resolution sampling also rare--gap into which UAS is only just allowing to be filled. \\
%Limitation for ocean studies, we did not study third dimension, volume (might be more relevant).

%Although limited sample size, insight into the scales at which science of ecology is making bulk of observations


%\paragraph*{Displayed math.} 
\bibliography{/Users/lestes/Dropbox/publications/fullbib}
\bibliographystyle{naturemag}

\noindent \textbf{Acknowledgements} This work was supported by funds from the Princeton Environmental Institute Grand Challenges program and the NASA New Investigator Program (NNX15AC64G). Erle Ellis, Jason Chang, and Labeeb Ahmed of the GLOBE Project (http://globe.umbc.edu) were supported by the U.S. National Science Foundation (1125210).



% Following is a new environment, {scilastnote}, that's defined in the
% preamble and that allows authors to add a reference at the end of the
% list that's not signaled in the text; such references are used in
% *Science* for acknowledgments of funding, help, etc.



% For your review copy (i.e., the file you initially send in for
% evaluation), you can use the {figure} environment and the
% \includegraphics command to stream your figures into the text, placing
% all figures at the end.  For the final, revised manuscript for
% acceptance and production, however, PostScript or other graphics
% should not be streamed into your compliled file.  Instead, set
% captions as simple paragraphs (with a \noindent tag), setting them
% off from the rest of the text with a \clearpage as shown  below, and
% submit figures as separate files according to the Art Department's
% instructions.

%\begin{scilastnote}
%\item This work was supported by funds from the Princeton Environmental Institute Grand Challenges program and the NASA New Investigator Program (NNX15AC64G). Erle Ellis, Jason Chang, and Labeeb Ahmed of the GLOBE Project (http://globe.umbc.edu) were supported by the U.S. National Science Foundation (1125210).
%\end{scilastnote}

\clearpage


\end{document}




















