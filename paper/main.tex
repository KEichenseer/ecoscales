% Use only LaTeX2e, calling the article.cls class and 12-point type.

\documentclass[12pt]{article}

% Users of the {thebibliography} environment or BibTeX should use the
% scicite.sty package, downloadable from *Science* at
% www.sciencemag.org/about/authors/prep/TeX_help/ .
% This package should properly format in-text
% reference calls and reference-list numbers.

\usepackage{scicite}
\usepackage{times}
\usepackage{graphicx}
\usepackage{lineno}

% The following parameters seem to provide a reasonable page setup.

\topmargin 0.0cm
\oddsidemargin 0.2cm
\textwidth 16.5cm 
\textheight 21cm
\footskip 1.0cm


%The next command sets up an environment for the abstract to your paper.

\newenvironment{sciabstract}{%
\begin{quote} \bf}
{\end{quote}}


% If your reference list includes text notes as well as references,
% include the following line; otherwise, comment it out.

\renewcommand\refname{References and Notes}

% The following lines set up an environment for the last note in the
% reference list, which commonly includes acknowledgments of funding,
% help, etc.  It's intended for users of BibTeX or the {thebibliography}
% environment.  Users who are hand-coding their references at the end
% using a list environment such as {enumerate} can simply add another
% item at the end, and it will be numbered automatically.

\newcounter{lastnote}
\newenvironment{scilastnote}{%
\setcounter{lastnote}{\value{enumiv}}%
\addtocounter{lastnote}{+1}%
\begin{list}%
{\arabic{lastnote}.}
{\setlength{\leftmargin}{.22in}}
{\setlength{\labelsep}{.5em}}}
{\end{list}}


% Include your paper's title here

\title{The Spatial and Temporal Domains of Modern Ecology \\
OR Ecology's Spatio-temporal Domains\\
OR Space, Time, and Ecology \\
OR Ecology: still a small-scale, field-based discipline} 
 
\author
{Lyndon Estes$^{\ast1, 2}$, Labeeb Ahmed$^{3}$, Kelly Caylor$^{2}$, Jason Chang$^{3}$, \\
Jonathan Choi$^{4}$, Erle Ellis$^{3}$, Paul Elsen$^{4}$, and Tim Treur$^{4}$ \\
\\
\normalsize{$^{1}$Woodrow Wilson School, Princeton University, Princeton, NJ 08544, USA}\\
\normalsize{$^{2}$Civil and Environmental Engineering, Princeton University, Princeton, NJ 08544, USA}\\
\normalsize{$^{3}$Geography and Environmental Systems, University of Maryland Baltimore County,}\\
\normalsize{Baltimore, MD 21250, USA}\\
\normalsize{$^{4}$Ecology and Evolutionary Biology, Princeton University, Princeton, NJ 08544, USA}\\
\\
\normalsize{$^\ast$To whom correspondence should be addressed; E-mail:  lestes@princeton.edu.}
}
% Include the date command, but leave its argument blank.

\date{}



%%%%%%%%%%%%%%%%% END OF PREAMBLE %%%%%%%%%%%%%%%%



\begin{document} 

% Double-space the manuscript.

\baselineskip24pt

% Make the title.

\maketitle 



% Place your abstract within the special {sciabstract} environment.

\begin{sciabstract}
A pithy and highly compelling abstract
%  An example of the style is the special\texttt{\{sciabstract\}} environment used to set up the abstract you see here.
\end{sciabstract}

\linenumbers
The scales at which ecosystems are observed plays a critical role in shaping our understanding of how they are structured and function \cite{levin_problem_1992,chave_problem_2013}.  Ecological patterns emerge within temporal and spatial domains that may be coarser or finer than the processes that shape them, which means that investigation across multiple scales is the \emph{sine qua non} for understanding ecological phenomena \cite{levin_problem_1992}. Awareness of the importance of scale has grown rapidly since the 1980s, accelerated by the need to understand how changes in the global climate, ocean, and land systems are affecting everything from individual populations (e.g. cite) to entire biomes (e.g. cite), while technological advances in areas such as remote sensing and genetics are making it ever-easier to quantify ecological features across a broad range of scales \cite{schneider_rise_2001,chave_problem_2013}.  

Given the importance of multi-scale studies for providing ecological understanding, and the growing ability to undertake them, it is important to rigorously assess whether ecology is becoming a multi-scale discipline. One approach to answering this question is to quantify the spatial and temporal domains within which observations in ecological studies are collected. Observations provide the necessary means for developing and testing the models that explain why ecological patterns vary in time and space \cite{levin_problem_1992,tilman_ecological_1989}, thus it stands to reason that the temporal and spatial range of ecological observations, and their density within different portions of those ranges, will shed light on modern ecology's progress towards a holistic, predictive understanding of ecosystems \cite{chave_problem_2013,levin_problem_1992}. In this study, we quantified the spatio-temporal domain of current ecological studies, using a representative sample of papers published between 2004-2014 in the top 30 ecological journals (by 2014 impact factor) to measure two key dimensions of spatial observation, resolution (grain) and total spatial extent, and their temporal corollaries, sampling interval and total temporal duration. We collected this information from 367 ecological observations (defined here as data collected from non-experimentally manipulated, or ``natural'' \cite{tilman_ecological_1989}, systems) reported within a 148 paper subset of 299 randomly selected articles (1.4\% of all papers). 

%16 + 29 + 26 + 67 + 14 + 37 + 10
In terms of spatial resolution, here defined as the two-dimensional space in which all measurable features of a natural object were recorded (as opposed to sub-sampled), the majority 63\% were collected in plots having resolutions $<$1 m$^2$, while 25\% were collected within plots of 1 m$^2$ up to 1 ha, and the remaining 12\% in plots of $\geq$1 ha (Fig. 1A). The total spatial extent covered (the number of sampled sites multiplied by the spatial resolution) by 85\% of observations was $<$10 ha, while 31\% covered less than 1 m$^2$ (Fig. 1B).  Only 7\% covered an extent $\geq$10 ha, with just 1.1\% spanning areas $\geq$10 million ha. 

\begin{figure}[!ht]
%\begin{wrapfigure}{c}{1\textwidth}
\includegraphics[width=1\textwidth]{figures/hists.pdf}
\vspace{-0.15 cm}
\caption{Histograms of the spatial resolution (A) and extent (B), sampling interval (C) and temporal duration (D) of ecological observations collected from the surveyed ecological studies. Bars represent percentage of the 367 collected records falling within each bin.}
\label{afoto1}
\end{figure}

In the temporal dimensions, 30\% of the assessed observations were ``once-offs'' that were not repeated (Fig. 1C). High frequency observations (ranging between as or more frequent than once per second up to daily) comprised 25\% of observations, 20\% were made at daily to monthly time steps, while 35\% and 4\% were respectively made at monthly up to yearly and yearly to decadal intervals.  The temporal duration of studies--the total amount of time the ecological feature was observed (the number of repeat observations multiplied by the effective sampling duration, SI)--was less than 1 day for 59\% of sampled observations,  35\% between 1 day and 1 year, and just 6\% covering greater than 1 year (including several paleoecological studies covering centuries to millenia; Fig. 1D).

The juxtaposition of these observational scales provides further insight into the spatio-temporal distribution of ecological observations and the domains in which they are concentrated (illustrated using two-dimensional kernel density estimates of the log-transformed scale values; Fig. 2). A comparison between the sampling interval and spatial resolution of ecological observations reveals three primary regions of observational concentration (Fig. 2A), the greatest being once-off samples with resolutions of 100 cm$^2$ to 1 m$^2$, followed next by observations of daily to monthly frequency of slightly finer resolution domain (10 cm$^2$ to slightly less than 1 m$^2$), and then a smaller center of monthly to yearly measurements collected at scales of 0.1-10 ha.  Almost no observations were collected with intervals greater than a decade, at all spatial resolutions, while high frequency (daily to sub-daily) observations were rare and collected at very fine resolutions ($<$1000 cm$^2$). Almost no observations with $\geq$1 m$^2$ resolution were collected with sub-second (effectively continuous) to hourly frequency, and the density of 10-1000 m$^2$ resolution, sub-daily to sub-monthly frequency observations was also low. 

\begin{figure}[!ht]
%\begin{wrapfigure}{c}{1\textwidth}
\includegraphics[width=1\textwidth]{figures/kde4.pdf}
\vspace{-0.15 cm}
\caption{Two-dimensional kernel density estimates of observational densities within the domains defined by A) sampling interval and spatial resolution, B) temporal duration and spatial extent, C) spatial resolution and spatial extent, and D) sampling interval and temporal duration. Density estimates were applied to the log-transformed values of each observation dimension, and density estimates are rescaled to represent percentages. }
\label{afoto1}
\end{figure}

In terms of temporal duration and spatial extent (Fig. 2B), the highest frequency ecological observations span time periods of 1 hour to $<$1 month, and cover $\leq$1 m$^2$ of space. The duration-extent domain extent appears to reach its apex at one day of duration and 1000-10,000 ha of extent. The very few observations collected that cover $>$1,000,000 ha were of sub-second duration, as these were collected by satellites that make instantaneous measurements.  

Comparing the two spatial scales of measurements against one another shows that observational extent scales log-linearly with resolution (Fig. 2C), revealing a nearly 1:1 correspondence up to 100 ha of both resolution and extent. This relationship indicates that the larger the plot size, the fewer the number of replicates, which is a characteristic of field-based studies because effort and cost typically increases with plot size \cite{kareiva_spatial_1988}. In our sample, 85\% of the observations were collected using manual, field-based methods. These stand in contrast to remotely sensed observations, which made up just 4\% of the sample, but which are characterized by plot resolutions that are very high relative to their spatial extent--these are the low density patches above the 1:1 line defined by the denser concentration of field studies, namely the patches at the following resolution/extent domains; 1-10 m$^2$/100-10,000 ha; 0.1-10 ha/10 million ha; 10-10,000 ha/100,000-10 billion ha.  

Comparing sampling interval to temporal duration shows that the bulk of once-off observations last between a minute to a week in duration (Fig. 2D). For repeated observations, the greatest density of observations were those made at monthly to yearly intervals and an aggregate duration of hours to weeks.  Very few repeat observations exceeded a total duration measured in years, and those that did were of daily to yearly frequency.  

This analysis reveals that ecology is still primarily a field-based science that is necessarily focused on small spatial scales, both in terms of sampling resolution and extent of coverage (Fig. 2C). The reason for this is primarily due to the surprisingly low use of remote sensing, which has been available for several decades and is increasingly easy to use. Yet, despite prominent and repeated calls for ecologists to adopt remote sensing because it provides a synoptic view that field measurements cannot \cite{turner_remote_2003,kerr_space_2003,pettorelli_satellite_2014}, and subsequent demonstration of its importance for multi-scale ecological studies \cite{estes_habitat_2008,estes_predictive_2011}, our results indicate that this observational method is dramatically under-utilized by ecologists, likely because of perceived (and in some cases real) logistical barriers \cite{pettorelli_satellite_2014}. [Something here about how this is falling; and in something about UAVs]

These relatively narrow spatial domains of observation naturally raise a question about the degree to which ecologists have truly set about to understand phenomena across multiple scales \cite{levin_problem_1992}.  Adding to this concern is the fact that the scales of observation we measured were not clearly reported in the majority of studies; the spatial resolution and extent of studies was uncertain in 64\% and 70\% of sampled cases (and even the number of spatial replicates was not clearly reported for 23\% of the sample), while uncertainty in the precise interval and study duration were uncertain in 43\% and 86\% of cases, respectively.  Although sensitivity analyses show that this uncertainty does not alter the domains of observation (SI), this lack of clarity supports the idea that ecology as a discipline is still insufficiently attentive to the concept of scale \cite{chave_problem_2013,wheatley_factors_2009}.  

However, in the temporal domains there appears to a greater diversity of scales represented (Fig. 1D). At least 70\% of sampled observations involved repeat assessment, and one-quarter of these were of high frequency. Automated surface-based sensors provided most of the very high frequency observations, and comprised 9\% of sampled observations, which shows that, unlike remote sensing, these technologies are being more broadly adopted in ecology. As with spatial resolution, finely resolved temporal data can be used to derive less frequent, aggregated observations (whereas long-interval data cannot be disaggregated to create high frequency observations), thereby enabling multi-scale analyses. 

Temporal duration was of course relatively narrow in most studies, but here we calculated the summed time of repeat observations, which is brief in most instances (e.g. for satellite imaging) and thus amounts to a short period of time. Frequent snapshots in time may be sufficient to capture the temporal dynamics of many phenomena (e.g. changes in vegetation cover \cite{hansen_high-resolution_2013}), and for these our measurement of this domain may understate the temporal breadth of ecological observations [note: might need to show this, at least in supplementals]. However, for more dynamic phenomena long periods of continuous observation may be more important for understanding dynamics than frequent repeats (e.g. behavioral ecology, can someone help here with a CITE and or a better example). For this reason we chose to capture the integrated observational duration, rather then the span of time between the first and last observation. 

[still need a wrap-up paragraph, e.g. ecology appears to not be producing the diversity of scales in observation that would allow it to develop and test hypothesis across scales, more so in spatial domains then in temporal. Technological advances are offering ever more opportunities to advance this, e.g. UAVs for high frequency, high resolution observation (Fig. 2.A), cite Gaston paper in Frontiers, Dandois and Ellis, but the question is will discipline overcome the inertia for adopting in particular geospatial advances, etc, etc.]

%Based on this analysis, ecology thus appears to have

%High frequency, high resolution sampling also rare--gap into which UAS is only just allowing to be filled. \\
%Limitation for ocean studies, we did not study third dimension, volume (might be more relevant).

%Although limited sample size, insight into the scales at which science of ecology is making bulk of observations


%\paragraph*{Displayed math.} 

\bibliography{/Users/lestes/Dropbox/publications/fullbib}

\bibliographystyle{Science}



% Following is a new environment, {scilastnote}, that's defined in the
% preamble and that allows authors to add a reference at the end of the
% list that's not signaled in the text; such references are used in
% *Science* for acknowledgments of funding, help, etc.

%\begin{scilastnote}
%\item We've included in the template file \texttt{scifile.tex} a new
%environment, \texttt{\{scilastnote\}}, that generates a numbered final
%citation without a corresponding signal in the text.  This environment
%can be used to generate a final numbered reference containing
%acknowledgments, sources of funding, and the like, per {\it Science\/}
%style.
%\end{scilastnote}


% For your review copy (i.e., the file you initially send in for
% evaluation), you can use the {figure} environment and the
% \includegraphics command to stream your figures into the text, placing
% all figures at the end.  For the final, revised manuscript for
% acceptance and production, however, PostScript or other graphics
% should not be streamed into your compliled file.  Instead, set
% captions as simple paragraphs (with a \noindent tag), setting them
% off from the rest of the text with a \clearpage as shown  below, and
% submit figures as separate files according to the Art Department's
% instructions.


\clearpage


\end{document}




















